\section{Introduction}

Some good intro here.

% When the internet was first established, everything sent over it was public. If A sent a message to B, anyone on their path through the internet could see that such a message was sent, and read the actual message. As of today, many messaging services are end-to-end encrypted, meaning that no one can read the contents of messages. However, for sufficiently powerful adversaries, it is still possible to find out who is talking to whom. The goal of anonymous communication is to hide this metadata: we want to create a system where A and B can send messages to each other over an untrusted network, without anyone knowing that they talk to each other. Designing such a system would not only be theoretically interesting, but could also be used to protect privacy in the real world; for further motivation to the problem, we refer to \cite{arvid}.

% Arguably, the most successful anonymous communication system is Tor \cite{dingledine2004tor}. It is being used by 2 million people every day \cite{torusage}, all of whom enjoy some level of privacy compared to the normal internet. However, Tor provides no privacy against an attacker that can see certain parts of the network: a global observer of the network can trivially figure out who sends data where, and even a weaker adversary can make inferences on communication patterns based on timing data. In this work, I will therefore focus on systems that have provable anonymity guarantees. There has been a lot of research on these systems, but no solution that is efficient enough to serve as a metadata-hiding replacement for Signal.
