\section{Purpose}
Previous MPM papers have focused on the security of a multi-round MPM where each client sends and receives exactly one message per round. This is not sufficient for several reasons.
\begin{itemize}
    \item Addra, the scheme we are basing on, is only proven secure when users hold exactly one conversation at a time. In our application, Clients may hold many different conversations at the same time. If not dealt with carefully, this can cause serious security problems, e.g. CF attacks. 
    \item As clients have different level of resources, running synchronous rounds is not economical. For example, big companies might wish rounds run faster to receive timely updates, while individual clients might not want to participate in each round to preserve bandwidth. We need to make sure security holds even if different users customize their apps for different settings.
    \item Practical messaging apps need many more components beyond sending and receiving messages, such as trust establishment, etc. These components will be constantly updated. It is important that the security of the MPM system is preserved by these components.
\end{itemize}

%\arvid{Pung and Talek both kinda handle multiple rounds no?}
%\stzh{Ah I made a mistake. Let me reword this more carefully.}

We design a formal security definition for a multi-round real-time messaging scheme. This draft is more based on the security definition presented in NIAR. 

\todo{Currently, we assume all clients have registered before execution. We also do not handle trust establishment.}

\todo{Also, CF attack is a huge problem! What should we do about it?}

\subsection{Conventions}
Double column makes the paper easier to read, but prevents me from writing writing very long formulas. Therefore, I am introducing a few conventions to avoid introducing long formulas.
\begin{itemize}
    \item When I write $f(\cdot)$, the dot might hide several variables.
    
    \item Given an oracle $O(x, \cdot)$ and a series $\{x_i\}_i$, define $O(\{x_i\}_i, \cdot)$ as the oracle whose input takes an extra argument $j$ and outputs $O(\{x_i\}_i, j, \cdot) = O(x_j, \cdot)$.

    \item When we say two experiments are indistinguishable, we mean the view of the adversary in the two experiments are indistinguishable. The view of the adversary consists of all inputs, outputs, and internal randomness of the adversary.
\end{itemize}