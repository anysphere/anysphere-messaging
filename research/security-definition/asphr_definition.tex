\newcommand{\frienddb}{\mathsf{frdb}}
\newcommand{\seqs}{\mathsf{seqstart}}
\newcommand{\seqe}{\mathsf{seqend}}
\newcommand{\seqr}{\mathsf{seqreceived}}
\newcommand{\inb}{\mathsf{in}}
\newcommand{\outb}{\mathsf{out}}
\newcommand{\msgdb}{\mathsf{msgdb}}
\newcommand{\msglb}{\mathsf{msglb}}
\newcommand{\ackdb}{\mathsf{ackdb}}
\newcommand{\trans}{\mathsf{trans}}
\newcommand{\ACK}{\mathsf{ACK}}
\newcommand{\ack}{\mathsf{ack}}


\section{Definition of Anysphere}
In this section, we define the Anysphere core protocol described in our whitepaper. We rigorously define how our core protocol is able to handle multiple contacts, run asynchronous rounds. In the next section, we will prove that our core protocol maintains correctness and security.
\subsection{A weaker Security Definition}
Unfortunately, Anysphere does not support the strongest security notion described above, due to CF attacks. For the threat model in our whitepaper, we assume that no friends are compromised. We now define a weaker security notion with this additional hypothesis.

\arvid{Another path to explore here would be to create a new leak function, say LeakCF, which contains the exact information leaked for the CF attack. For example, one potentially useful leak function would be one that contains the current number of friends, or perhaps the number of friends but rounded to the nearest 10. In the worst case (such as our current prioritization case), we would leak the timing that a message actually gets sent to the server, which might be very hard to model here... Hmmmm}

\begin{definition}[SIM-secure with no compromised friends and bounded friends]
We say that a set of input $\{\cI_{i, t}\}_{i \in \cH, t \in [T]}$ satisfy \textbf{no compromised friends} if for any $i \in \cH, j \in \cK$ and $t \in [T]$, we have
$$\trust(\reg_j) \neq \cI_{i, t}.$$
We say that a set of input $\{\cI_{i, t}\}_{i \in \cH, t \in [T]}$ satisfy \textbf{$B$-bounded friends} if for any $i \in \cH$, the set
$$\{\reg: \exists t, \trust(\reg) = \cI_{i, t}\}$$
has cardinality at most $B$.

We say a messaging scheme is correct with $B$-bounded friends iff for any parameters $(\lambda, N, T, \{\cI_{i, t}\})$ of the honest server experiment, if $\{\cI_{i, t}\}$ satisfy $B$-bounded friends, then the scheme produces the correct views.

We say a messaging scheme is SIM-secure with no compromised friends / $B$-bounded friends iff for any polynomial upper bounds on $N$ and $T$, there exists a p.p.t simulator $\Sim$ such that for any p.p.t adversary $\cA$, the view of $\cA$ is indistinguishable in $\mathsf{Real}^{\cA}(1^{\lambda})$ and $\mathsf{Ideal}^{\cA, \mathsf{\Sim}}(1^{\lambda})$, provided that the input set $\{\cI_{i, t}\}_{i \in \cH, t \in [T]}$ satisfies no compromised friends / $B$-bounded friends.
\end{definition}
\textbf{Remark}: It is possible to bypass CF attacks under either $B$-bounded friends or no compromised friends model. We'll assume the $B$-bounded friends for now.

Currently we have $B = 2$.

We now explain the construction of Anysphere. 
\subsection{Cryptography Primitives}
\todo{Are there existing definition of these schemes?}
Anysphere relies on two crypto primitives: a symmetric key AE, and a PIR scheme. We outline formal simulator-based definition of the two.
\subsubsection{Key Private IND-CPA symmetric key AE scheme}
\arvid{we probably want IND-CCA to guarantee integrity?}
A key-private IND-CPA symmetric key AE(Authenticated Encryption) scheme is a quadruple of algorithms $(\gen, \mathsf{KX}, \enc, \dec)$ with specifications
\begin{itemize}
    \item $\gen(1^{\lambda}) \to (kx^P, kx^S)$,
    \item $\mathsf{KX}(1^{\lambda}, kx_A^P, kx_B^S) \to sk_{AB},$
    \item $\enc(sk_{AB}, m) \to ct,$
    \item $\dec(sk_{BA}, ct) \to m.$
\end{itemize}
such that 
\begin{definition}[Correctness]
\label{defn:KX-Correctness}
If we let
\begin{enumerate}
    \item $(kx^P_A, kx_A^S), (kx^P_B, kx_B^S)  \leftarrow \gen(1^{\lambda}).$
    \item $sk_{AB} \leftarrow \mathsf{KX}(1^{\lambda}, kx_A^P, kx_B^S), sk_{BA} \leftarrow \mathsf{KX}(1^{\lambda}, kx_B^P, kx_A^S)$
\end{enumerate}
then for any plaintext $m$, we have
$$\dec(sk_{BA}, \enc(sk_{AB}, m)) = m.$$
Furthermore, for any secret key $sk' \neq sk_{AB}$, we have
$$\dec(sk_{BA}, \enc(sk', m)) = \bot.$$
\todo{negligible probability here?}
\end{definition}
\begin{definition}[Key private IND-CPA]
Let $N, R$ be polynomial in $\lambda$. Consider two experiments.
\begin{figure}[h!]
\begin{framed}
\textbf{Real World Experiment}
\begin{enumerate}
    \item For $i$ from $1$ to $N$, $(kx_i^P, kx_i^S) \leftarrow \gen(1^{\lambda}).$
    \item For $(i, j)$ in $[N]^2$, $sk_{ij} \leftarrow \mathsf{KX}(1^{\lambda}, kx_i^P, kx_j^S)$.
    \item For $r$ from $1$ to $R$
    \begin{enumerate}
        \item $i, j, m \leftarrow \cA(1^{\lambda}, r, \{kx_i^P\}_{i \in [N]}, \{ct_{s}^0\}_{s < r}\})$
        \item $ct^{0}_r \leftarrow \enc(sk_{ij}, m)$.
    \end{enumerate}
\end{enumerate}
\textbf{Ideal World Experiment}
\begin{enumerate}
    \item For $i$ from $1$ to $N$, $(kx_i^P, kx_i^S) \leftarrow \gen(1^{\lambda}).$
    \item For $(i, j)$ in $[N]^2$, $sk_{ij} \leftarrow \mathsf{KX}(1^{\lambda}, kx_i^P, kx_j^S)$.
    \item For $r$ from $1$ to $R$
    \begin{enumerate}
        \item $i, j, m \leftarrow \cA(1^{\lambda}, r, \{kx_i^P\}_{i \in [N]})$.
        \item $ct^1_r \leftarrow \Sim(\{kx_i^P\}_{i \in [N]})$.
    \end{enumerate}
\end{enumerate}
\end{framed}
\end{figure}
Then we say the encryption scheme is key-private IND-CPA if there exists a p.p.t simulator $\Sim$ such that for any p.p.t adversary $\cA$, the view of $\cA$ under the real world experiment and the ideal world experiment are computationally indistinguishable. The view of $\cA$ is defined as the input, output, and random coins of $\cA$, plus the array $\{ct_r\}_{r \in [R]}$.
\end{definition}
\begin{definition}[EUF-CMA Authentication]
Let $kx^P_A, kx^P_B, \sk_{AB}, \sk_{BA}$ be the same as \cref{defn:KX-Correctness}. For any p.p.t. adversary $\cA^O$ with access to oracle $O$, if
$$m \leftarrow \cA^{\Enc(\sk_{AB}, \cdot), \Enc(\sk_{BA}, \cdot)}(1^{\lambda}, kx^P_A, kx^P_B)$$
then 
$$\PP(m \notin Q, \Dec(\sk_{BA}, m) \neq \bot) = \negl(\lambda),$$
where $Q$ is the set of messages that $\cA$ queried to the oracle.
\end{definition}

\subsubsection{PIR Protocol}
Central to our application is the Private Information Retrieval(PIR) Protocol. It consists of four efficient algorithms
\begin{itemize}
    \item $\gen(1^{\lambda}, n) \to (pk, sk)$.
    \item $\query(1^{\lambda}, sk, i) \to ct$.
    \item $\answer^{DB}(1^{\lambda}, pk, ct) \to a$.
    \item $\dec(1^{\lambda}, sk, a) \to x_i$.
\end{itemize}
% \arvid{Another PIR syntax, which I believe might be more standard?}
% \begin{itemize}
%     \item $\query_n(1^{\lambda}, i) \to (q, \mathsf{st})$.
%     \item $\answer_n(1^{\lambda}, D, q) \to a$.
%     \item $\dec_n(1^{\lambda}, \mathsf{st}, a) \to d$.
% \end{itemize}
% Nevermind! We are recycling the secret key for the PIR queries, because generating the keys is computationally expensive, and it does not affect security (it just becomes slightly less clean)
where $D$ is a length $n$ database with int64 entries, and $n$ is upper bounded by some polynomial $n(\lambda)$. It must satisfy
\begin{definition}[Correctness]
For any database $DB$ of length $n$ with int64 entries, and any message $ct$ of length $L_{\msg}$, if we run
\begin{enumerate}
    \item $(pk, sk) \leftarrow \gen(1^{\lambda}, n)$.
    \item $ct \leftarrow \query(1^{\lambda}, sk, i)$.
    \item $a \leftarrow \answer^{DB}(1^{\lambda}, pk, ct)$.
    \item $x_i \leftarrow \dec(1^{\lambda}, sk, a)$.
\end{enumerate}
then $x_i = DB[i]$.
\end{definition}
\begin{definition}[Privacy]
for any $\lambda, n$, consider the following two experiments. Then we say the PIR scheme is \textbf{secure} if there exists a p.p.t simulator $\Sim$ such that for any p.p.t adversary $\cA$, the view of $\cA$ under the real world experiment and the ideal world experiment are computationally indistinguishable. The view of $\cA$ is defined as the input, output, and random coins of $\cA$, plus the query $ct^b$.
\begin{figure}[h!]
\begin{framed}
\textbf{Real World Experiment}
\begin{enumerate}
    \item $(pk, sk) \leftarrow \gen(1^{\lambda}, n).$
    \item $i \leftarrow \cA(1^{\lambda}, n, pk)$.
    \item $ct^0 \leftarrow \query(1^{\lambda}, sk, i)$.
\end{enumerate}
\textbf{Ideal World Experiment}
\begin{enumerate}
    \item $(pk, sk) \leftarrow \gen(1^{\lambda}).$
    \item $i \leftarrow \cA(1^{\lambda}, n, pk)$.
    \item $ct^1 \leftarrow \Sim(1^{\lambda}, n, pk)$.
\end{enumerate}
\end{framed}
\end{figure}
\end{definition}
For our implementation, we use libsodium's key exchange functionality for the $\mathsf{KX}$ function, then libsodium's secret key AEAD for the $\enc$ and $\dec$ functions. This provides a Key Private IND-CPA symmetric key encryption scheme. We use Addra as the PIR protocol.
\subsection{Messages, Sequence numbers, ACK}
To guarantee the consistent prefix property, each client $i$ labels all messages to be sent to another client $j$ with a sequence number. In the order client $i$ receives $\send$ instructions to client $j$, client $i$ label the messages with sequence number $1,2,\cdots$.

Critical to both consistent prefix and eventual consistency is the ACK messages. An ACK message is a special type of messages denoted $\ACK(k)$.\todo{In the whitepaper an additional chunk number is used. Since this definition does not use chunking, I'll ignore it}. When client $j$ sends $\ACK(k)$ to client $i$, it means ``I have read all messages up to message $k$ from you". As we will soon define rigorously, user $i$ will keep broadcasting message $k$ until user $j$ sends $\ACK(k)$, in which case they begin broadcasting message $k + 1$.

Because the client need to keep track of these metadatas, the client labels each message the user inputs with the time and the sequence number. If the user inputs a message $\msg$ of length $L_{\msg}$, the client will transmit the labeled message\footnote{Called chunk in the code.} $\msg^{lb} = (k, \msg)$, where $k$ is the sequence number and $\msg$ is the actual message. Let $L_{\msglb}$ denote the length of the labeled message, and let $L_{\ct}$ be the length of the ciphertext generated by encoding $\msg^{lb}$ with $\Pi_{\sym}.$. 

We will take $L = L_{\ct}$ for the PIR scheme.
\subsection{The Anysphere Core Protocol}
\todo{I'm going to stick to our actual implementation as closely as possible.}
Let's first recap and summarize what we have defined so far. Recall that
\begin{enumerate}
    \item $\lambda$ is the security parameter.
    \item $N$ is the number of users.
    \item $T$ is the number of timesteps our protocol is run.
    \item $L_{\msg}$ is the length of each message we send.
    \item We use a key-private IND-CPA symmetric key AE scheme $\Pi_{\sym}$.
    \item We use a PIR protocol $\Pi_{\pir}$.
\end{enumerate}
We can now formally define Anysphere's core protocol.
\begin{definition}[The Anysphere Core Protocol]
Our protocol, $\Pi_{\asphr}$, implements the methods of \cref{def:messaging-scheme} as below. In each method, the caller stores all inputs for future use.
\vspace{10pt}

$\mathbf{\Pi_{\asphr}.C.\mathsf{Register}}(1^{\lambda}, i, N)$
\vspace{5pt}
\hrule
\vspace{5pt}
\begin{enumerate}
    \item Initialize map $\frienddb$. The map take registration info as keys, and the following fields as values.
    \begin{itemize}
        \item $\mathsf{sk}$, the secret key.
        \item $\seqs$, the sequence number of the current message being broadcasted to the friend.
        \item $\seqe$, the highest sequence number ever assigned to the friend.
        \item $\seqr$, the highest sequence number received from the friends.
    \end{itemize}
    \todo{These fields are currently implicit. Included for simplicity}
    \arvid{a bit confused between seqend and seqstart...}
    \stzh{in other words, the inbox contains messages in [seqstart, seqend].}
    \item Initialize maps $\inb, \outb$. The maps take registration info as keys, and arrays of messages as values.\footnote{They are named Friend, Inbox, Outbox in our code. Our code is slightly more complicated to support features like sending to multiple friends and chunking.}
    \item Set a transmission schedule $T_{\trans}$. The user can customize this parameter.
    \item $(kx^P, kx^S) \leftarrow \gen(1^{\lambda})$. 
    \item Initialize PIR Keys. $(\pk_{\pir}, \sk_{\pir}) \leftarrow \Pi_{\pir}.\gen(1^{\lambda}, N).$
    \item Return $\reg \leftarrow (i, kx^P)$.
\end{enumerate}
\vspace{10pt}
$\mathbf{\Pi_{\asphr}.S.\mathsf{InitServer}}(1^{\lambda}, N)$.
\vspace{5pt}
\hrule
\vspace{5pt}
Initialize arrays $\msgdb, \ackdb$ of length $N$. Fill them with random strings.

\vspace{10pt}
$\mathbf{\Pi_{\asphr}.C.\mathsf{Input}}(t, \cI)$
\vspace{5pt}
\hrule
\vspace{5pt}
This method runs in two phases. Phase 1 handles the user input $\cI$, and phase 2 formulates the server request.

Phase 1: 

If $\cI = \emptyset$, do nothing. 

If $\cI = \send(\reg, \msg)$, 

\begin{enumerate}
    \item Check if $\reg$ is in $\frienddb$. If not, break.
    \item Add $1$ to $\frienddb[\reg].\seqe$. 
    \item Push $(\frienddb[\reg].\seqe, t, \msg)$ to $\outb[\reg]$.
\end{enumerate}

If $\cI = \trust(\reg)$.
\begin{enumerate}
    \item Check if $\reg$ is in $\frienddb$. If so, break.
    \item $(i, kx_f^P) \leftarrow \reg$.
    \item $sk \leftarrow \Pi_{\sym}.\mathsf{KX}(1^{\lambda}, kx_f^P, kx^S).$
    \item $\frienddb[\reg] \leftarrow \{\sk: sk,  \seqs: 1, \seqe: 0, \seqr: 0\}$.
\end{enumerate}

Phase 2:
\begin{enumerate}
    \item If $t$ is not divisible by $T_{\trans}$, return $\emptyset$.
    \item Let $\{\reg_1, \cdots, \reg_k\}$ be the keys of $\frienddb$, with $k\leq B$. Construct $S = [\reg_1,\cdots, \reg_k, \reg, \cdots, \reg]$, where we add $B - k$ copies of $\reg$, our own registration info. Sample $\reg_s, \reg_r$ uniformly and independently at random from $S$. \arvid{this is not what we currently do. maybe we should. we should make a decision on the CF attack here and what is acceptable. i think my favorite idea is leaking a rounded version of the number of friends, or something like that. however, the way the code works now where we pick a random friend among the friends that we have outgoing messages to is quite nice because it means that messages will get delivered much faster (especially once we implement PIR batch retrieval)...}
    \item Let $\msg$ be the message with sequence number $\frienddb[\reg_s].\seqs$ in $\outb[\reg_s]$. If $\outb[\reg_s]$ is empty, let $\msg = 0^{L_{\msg}}$.
    \item $\sk \leftarrow \frienddb[\reg_s].\sk$.
    \item $\seqr \leftarrow \frienddb[\reg_s].\seqr$.
    \item Encrypt Messages with $\sk$.
    \begin{itemize}
        \item $\ct_{\msg} = \Pi_{\sym}.\enc(\sk, \msg).$
        \item $\ct_{\ack} = \Pi_{\sym}.\enc(\sk, \ACK(\seqr))$. \arvid{should we talk about the ACK db here, and the fact that we always send all ACKs to everyone? maybe we shouldn't do that anymore... it was necessary for prioritization, but if we don't want to do prioritization then maybe we shouldn't do it anymore}
    \end{itemize}
    \item Let $\reg_r = (i_r, \_)$. Formulate a PIR request for index $i_r$. %\arvid{this $r$ is not the same as the $r$ in $\reg_r$? maybe this should be $i_r$ or something}
    \begin{itemize}
        \item $\ct_{\query} \leftarrow \Pi_{\pir}.\query(1^{\lambda}, \sk_{\pir}, i_r).$
    \end{itemize}
    \item return $\req = (\ct_{\msg}, \ct_{\ack}, \pk_{\pir}, \ct_{\query})$.
    \item Remember $\reg_r$.
\end{enumerate}
\vspace{10pt}
$\mathbf{\Pi_{\asphr}.S.\mathsf{ClientRPC}}(t, \{\req_i\}_{i = 1}^N)$
\vspace{5pt}
\hrule
\vspace{5pt}
\item for $i$ from $1$ to $N$
\arvid{do we want to deal with authentication tokens here? only if we modify the security definition to include potentially malicious clients, which I'm not sure is worth the trouble...}
\stzh{I vote for no Authentication token for now.}
\begin{enumerate}
    \item If $\req_i = \emptyset$, continue. Parse $\req_i = (\ct_{\msg}, \ct_{\ack}, \pk_{\pir}, \ct_{\query})$.
    \item $\msgdb[i] \leftarrow \ct_{\msg}, \ackdb[i] \leftarrow \ct_{\ack}$.
    \item $a_{\msg} \leftarrow \Pi_{\pir}.\answer^{\msgdb}(1^{\lambda}, \pk_{\pir}, \ct_{\query}).$
    \item $a_{\ack} \leftarrow \Pi_{\pir}.\answer^{\ackdb}(1^{\lambda}, \pk_{\pir}, \ct_{\query}).$
    \item $\resp_i \leftarrow (a_{\msg}, a_{\ack}).$
\end{enumerate}
\vspace{10pt}
$\mathbf{\Pi_{\asphr}.C.\mathsf{ServerRPC}}(t, \resp)$
\vspace{5pt}
\hrule
\vspace{5pt}
\begin{enumerate}
    \item Parse $\resp = (a_{\msg}, a_{\ack})$. Let $\reg_r, \sk_{\pir}$ be defined in the last call to $\Pi_{\asphr}.C.\mathsf{Input}$.
    \item $\ct_{\msg} \leftarrow \Pi_{\pir}.\Dec(1^{\lambda}, \sk_{\pir}, a_{\msg}).$
    \item $\ct_{\ack} \leftarrow \Pi_{\pir}.\Dec(1^{\lambda}, \sk_{\pir}, a_{\ack}).$
    \item $\sk \leftarrow \frienddb[\reg_r].\sk$.
    \item Decipher the message.
    \begin{enumerate}
        \item $\msg^{lb} \leftarrow \Pi_{\sym}.\Dec(1^{\lambda}, \sk, \ct_{\msg})$.
        \item If $\msg^{lb} = \bot$ or $\msg^{lb}[0]$ is not $\frienddb[\reg_r].\seqr + 1$, ignore the message.
        \item Add $1$ to $\frienddb[\reg_r].\seqr$. 
        \item Let $\msg$ be $\msg^{lb}[1]$. Push $\msg$ to $\inb[\reg_r]$.
    \end{enumerate}
    \item Decipher the ACK.
    \begin{enumerate}
        \item $\ack \leftarrow \Pi_{\sym}.\Dec(1^{\lambda}, \sk, \ct_{\ack})$.
        \item If $\ack = \bot$ or $\ack$ is not the form $\ACK(k)$ for some $k$, ignore the ack.
        \item Let $\ack = \ACK(k)$. If $k < \frienddb[\reg_r].\seqs$, ignore the ack.
        \item $\frienddb[\reg_r].\seqs \leftarrow k + 1$.
    \end{enumerate}
\end{enumerate}
\vspace{10pt}
$\mathbf{\Pi_{\asphr}.C.\mathsf{GetView}}()$
\vspace{5pt}
\hrule
\vspace{5pt}
Let $\cF$ be the set of keys in $\frienddb$. Let $\cM$ be the set $\{(\reg_r, \msg): \msg \in \inb[\reg_r]\}$. Return $(\cF, \cM)$.

\end{definition}