\subsection{The Impersonation Attack} 
We next describe an attack on the trust establishment system $\trust(\cdot)$. Due to a lack of central authorities, our system uses the registration information $\reg$ as the unique identifier of a client / user. The registration information is a tuple $\reg = (i, \kx^P)$ containing the index $i$ of the client in the PIR database, and the key exchange public key $\kx^P$ of the client. This information is intended to be posted on the client / user's social media, so that others can verify that the registration info indeed belongs to the person.

Suppose user $A, B$ are friends, and their registration informations are $\reg_A = (i_A, \kx^P_A)$ and $\reg_B = (i_B, \kx^P_B)$ respectively. Then a malicious user $C$ can declare on their webpage that their registration information is $\reg_C = (i_C, \kx^P_B)$. If user $A$ establishes trust with user $C$, then user $B$ will ``overhear" all conversations between user $A$ and user $C$, since the way $B$ decides if a message from $A$ is meant for them is by decrypting user $A$'s ciphertext with their shared public key $\sk_{AB}$ (see \cite[Figure 3]{whitepaper}). By letting $\sk_{AC} = \sk_{AB}$, user $C$ compromised the integrity of user $A$ and $B$'s conversation defined in \cref{defn:messaging-integrity}.

There are many ways to resolve this issue, such as adding a signature to the registration information, or labeling each message with their intended recipient. This attack is more of a subtlety in the proof, and motivates the Eval-security definition \cref{defn:AE-eval-security}.